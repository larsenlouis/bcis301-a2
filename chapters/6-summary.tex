%!TEX root = ../assignment2.tex

\section{Summary}

In this article, we have created a 3-year IT deployment plan for department of computing of Ara Institute of Canterbury. The functions in the department is first identified followed by a qualitative analysis that has found the issues relating to the functions. Then, possible solutions have been put forward as candidates into an evaluation framework. The selected solutions from the evaluation process finalizes the planning process before an hierarchy of IT deployment plan is built to illustrate the whole process.

To revise what have done structurally, here is the revision of each section.

In introduction, the goal of this article is defined as the 3-year IT deployment plan for Computing Department in Ara Institute of Canterbury. A brief backgorund information is provided in terms of the institute and the department. Finally, the topics to be covered are listed section by section at the end of Introduction.

In methodology, a method of plan making is defined which involves data collecting, data analysis, data evaluation and the ultimate hierarchy of plan. Functional analysis is used to recognize the functions inside the department, followed by a qualitative analysis which identifies and group the issues for the functions. Solutions are proposed to handling each topic of issues. A framework of evaluation is then given to select the candidate solutions. Deployment plan is presented by an hierarchy where the results of the goal of plan, data collection, data evaluation are included.

In data analysis, the functional analysis has found 12 functions in the computing department, followed by the qualitative analysis that has identified 32 facts with issues in 5 primary themes of issues, namely budget, communication, manual process, security and software. sub-topics of the primary themes are listed with solid samples.

In solution analysis, 6 candidate solutions are put forward to solve the primary topics of issues. Using the predefined evaluation framework in methodology, an evaluation process has been conducted to select the best options for the IT deployment plan. As a result, 4 of them have selected and surprisingly have covered all the topics of issues. The coverage assurance has not been designed in the methodology by the way.

In deployment plan, the plan is assembled in the hierarchy structure by all the elements that have produced in previous sections. The complete IT deployment plan is now officially delivered.

% In introduction, a brief information has been given, stating what we are investigating and what result we will produce. 3 keywords of project management, change management and risk management have been introduced to fund our further research. The methodology have covered how we do our qualitative analysis, how the evaluation is designed and how to make our own evaluation tool. A subsequent findings of Phase 1 has presented with the primary themes we identified with some samples of our findings in each theme. They are control, people, process and structure. We have also done the evaluation process for the proposals to select the solutions in this phase. As a result, 21 solutions have been selected. The findings of Phase 2 is structurally similar to the previous section, but with 10 case studies. The same primary themes have been identified as well. We have provided some samples in each primary themes as well. With the same evaluation process, we have selected 21 solutions in evaluation process. In outcome, we have analyzed the findings of both phases and found that there is a certain degree of agreement in the findings of the two phases. We have proceed to create our own evaluation tool with the predefined method in the methodology, which has produced a mark sheet for practitioners to use. In conclusion, we look back what we have done section by section.
