%!TEX root = ../assignment2.tex

\section{Data Analysis}

\subsection{Overview of Data}

Besides the interview scripts notes which were taken by Sarah last year, we also interviewed a student/teaching assistant and the head of the department two weeks ago. Though another student/teaching assistant is also planned to be interviewed by us, we have not had a chance to interview him for unknown reasons. So those scripts and both interviews with the student and the head of the department are assets we possess. Based on this information, we conducted a functional analysis and qualitative analysis which collect 32 facts across 12 functions.

\subsection{Functional Analysis}

Department of Computing is the department dedicated teaching ICT students, which provides courses of software engineering, networking, business analysis, cyber security. Three divisions, that is administrative support, Operational Manager (ICT as a profession), and Operational Manager (Community Computing) comprise this Department. From the name of these divisions, we can see that two main teaching fields which are IT as profession and communication computing are provided in this department.

\subsection{Primary Themes}
Five primary themes of the issues are identified from 21 secondary themes in our coding data, which were budget, communication, manual process, security, and software. Below are a few typical samples for each theme.

\subsubsection{Budget}
Budget is about the expenditure for a department in the annual reckoning. It is usually fixed but deficient since the unexpected expense happened casually. The insufficient budget was noticed through the research in the first phase. The interviewee from CISCO Network Academy informs us that they need more expensive switches and routers to extend the network being used. They need more space to accommodate students and equipment. They also want to attract more students to learn networking, and they need more money to carry out marketing activities. They actually meet difficulties in convincing Ara that paying more money to Cisco Network Academy is worthwhile enough. The budget issue is a realistic problem, while it is mainly a governance problem. Through deploying more Linux systems can alleviate the budget issues to some extent. Mostly, we cannot do more to resolve them by technical means. Therefore, we will not discuss this issue much in this article.

\subsubsection{Communication}
Communication has been recognized as a general issue for CISCO Network Academy and the Operations Manager. According to the interview with the tutor of CISCO Network Academy, the desire to maintain a better connection with previous students was expressed. He also pointed out that nothing was taught on teaching students how to create connections with industry except the "geeky stuff" in their Academy. For our students, it is important to own soft power besides their technical power to get competitive power in the talents market. How to create a safe and effective social community is one of the important issues to be resolved. The Operation Manager claimed that no one knows we are the biggest CISCO system trainer in New Zealand. In regards to the roles and responsibilities, there is a serious issue of lacking awareness and communication. Besides this complaint, the engagement with students is also defective, it needs to be improved by better communication tool and mechanics.

\subsubsection{Manual Process}
Manual Process is one of the most complained issues in this analysis. There are so many manual processes involved in almost all functions in the computing department. It is reported by all the interviewees, that is the head of the department, the student/teaching assistant, the tutor from Ethics, the tutor from Tech Labs, the tutor from software engineering, and the OP Manager. All of them have experienced various inconvenience from manual processes (mainly paperwork) in their daily work. Below are the details which are cataloged in functions for these inconveniences.

\paragraph{Administration}
The head of the department, as the most important person in the computing department, also feels pain on his administrative things since the internal process is manual. He has pointed out that changing something in course, such as network protocols, he had to fill in forms and send the forms to corresponding division, then wait for approvals. All the processes, including the monitoring process, are manual processes, which is lots of paperwork. As spectators, we can feel the pain when we review the interview scripts. The Operational Manager was also tired of too many manual processes when they perform "Moderation of assessments". "Every 5 years we have an assessment for every course, each course has an external assessment. It is entirely manual processes. All the things are in a spreadsheet; someone has to remember to do it; someone has to follow up if it has been done." The Operational Manager complained to us. "Techniques could do something to assist us, it also can reduce the risks." She said. In regards to the Re-Enrolment system, the same Manager hopes that if there is an IT system which can help to automate the whole process. Too many paper forms which were manually edited in Microsoft Word, it is annoying and easy to get something wrong, currently. finally, the tutor from Tech Labs suggests that purchase management is also paper-based. When they need to buy equipment or something else, they have to fill out quite a few paper forms for the IT department to review and purchase.

\paragraph{Timetabling}
The interviewee which is the OP manager claims that there is no specific IT program for timetabling, that she has to talk with individual tutors and manually find out which teaching sessions or hours are required, and that she has to manually put this information into a spreadsheet. Plus, the more distressing is that this is not the end. Once she completes the task on the spreadsheet, she has to manually put the timetable data into tribal software. She is tired of these things, it is not only boring, inefficient but also error-prone. 

\paragraph{CAPEX Requests}
A similar scene occurs in processing CAPEX requests. The same OP Manager is also in charge of processing CAPEX requests for the computing team. Manually filling in forms is required. Tracking the status of each CAPEX requests is an impossible task by human labor. Thus, there is no way to guarantee if the money has been correctly spent.

\paragraph{Attendance}
There is no unified automatic attendance system in the Computing department. Some teachers use class time to manually tick every student using the tribal application, which is a rather time-consuming thing in a larger class. Other teachers use a paper sheet, which allows students to sign on it. Sometimes a teacher or a teaching assistant may forget to pass the paper sheet to students, which eventually all the students being "absent". This situation has been observed and experienced several times by us in this semester. In fact, a conclusion can be made that both approaches are far from ideal. The tutor from the Software Engineering section argues that they have tried many alternatives but none of them is ideal.

\paragraph{Result System}
In regards to the results of the students, once a tutor confirmed the result for a certain student, he or she can immediately get his or her result, which is good. However, in case the tutor makes a mistake with a result, the tutor has to perform a manual process to change the result. Additionally, the result has to be extracted from the tribal database manually when a results meeting is held from some students who need additional support with their results. It nearly occurs at the end of every semester prior to the final results being confirmed. 

\paragraph{Ethics Applications}
The ethics tutor, who charges of managing the ethics in the computing department, claims that a student who needs ethics clearance has to fill out four ethics applications and that every application is paperwork. Filling out the paper form is not conducive to keeping the participants anonymous. It is much better if they are filled in an online form remotely instead of a paper form.

\paragraph{Booking Classrom}
Besides the functions we mentioned in the earlier paragraph, one of the teaching assistant representatives, she suggests there is no IT system to help them accomplish the assistant works. As a busy student to graduate, she has to fill in a paper timesheet to get paid. In addition, booking a classroom is also a manual process, which requires her to talk with the Operating Manager to get or to confirm a classroom, according to the interview.

\subsubsection{Security} 
Security is critical for any computer system, especially for those computer systems which is used for a public purpose. For security reasons, there are many restricts to the users when they access these computers in a computer lab. Security usually conflicts with convenience. Balancing security and convenience is the key to ensuring safety and ensuring user satisfaction. Below are some of the issues students and teachers encountered at a computer Lab.

\paragraph{Separated Networks}
Nowadays, the networking technique develops fast day by day. However, for historical and security reasons, we were told that the network in CISO Network Academy is physically separated from the Ara network. The separation of the two networks is important for security. On the other hand, it has brought much inconvenience to teachers and students. For instance, somethings can only do on tech labs network, If a student needs to access any content on the Ara network, he has to walk to an Ara computer lab to get those content then come back because the two networks are separated. It is no surprise that someone cannot access S disk or other shared drives on a certain computer. People know that this is because security reasons but inconveniences are inconveniences, there should be an approach can reach both the requirements from the Cisco Network Academy and Software Engineering. In terms of current status, both networks have security problems. According to one of the interviewees from Cisco Network Academy, around 50 pages security reports are generated for both tech labs network and Ara network annually, and the issues in the reports need to be fixed as soon as possible.

\paragraph{Permissions}
Computer labs are the most frequently visited place for the students in the Department of Computing. Unfortunately, it is also a place that students may encounter troubles. Considering the security of the operating system, a student cannot install new software onto the host operating system of the computer in computer labs. So students can only use the pre-installed software on these computers. We were told and experienced that the number and variety of pre-installed software are different. We often encounter that there is no Visio installed on this computer at this computer Lab, and there is no office on that computer at another computer Lab. The most astounding thing is that Git which is the most popular version control system is not installed on all machines either. Similarly, an interviewee from the Software Engineering division also complained that they have no permission to create virtual machines and cannot do certain things, such as install a new version of the development tools.

\subsubsection{Software} 

Through the scripts note from Sarah, the software student can use have to be approved by the IT department of Ara. This means teachers and students as the final user only have few choices with software. In fact, our interview and historic interview also proved this point. That is also the start point of the complaint from the interviewees. 

\paragraph{Virtualization}

According to the tutor from Cisco Network Academy, they are considering to move to cloud-based networks from purchasing new hardware. To accomplish this goal, a virtualized environment is expected. It means more computers which support to create virtualized devices are needed. Similarly, one of the software engineering tutors is also seeking a Docker like sandbox environment which can allow students to run applications with more freedom. 

\paragraph{Moodle}

Moodle is very popular in the Institutes and Universities in New Zealand. It is open source, free and powerful. It has been proved a good learning platform in the world although it has certain limitations. The tutor from Tech Labs was not satisfied with Moodle because of the terrible text editor, he or she has to use HTML code to make certain content looks better. As a teaching assistant, Sarah noticed that some parts of Moodle 
are not consistent with the others. Moodle has already been a requisite in the teaching process, so it will be better if certain functions such as booking classroom can integrate into Moodle. In fact, Moodle has a plugin system. The easiest and most maintainable way to add new functions to Moodle is by writing a plugin for Moodle.

\paragraph{Virtual Machines}

To support students running more software with more freedom, a virtual machine is a sensible choice. Comparing to a real computer, it has better compatibility with varied guest operating system. However, there has been existing a licensing issue for the Microsoft Products for a long time. "Licensing of software is not clear on virtual machines at all." said the Tech Labs tutor. The license system of Microsoft is particularly complex, even Microsoft New Zealand cannot answer how to be legal when using windows on a virtual machine. Ara needs to spend a month or two in the United State to work out the licensing issue. It is much money. At the same time, virtual machines are broadly used as a sandbox environment these days. There is a huge demand for using virtual machines in Ara, especially in the computing department. Students and teachers can easily create and destroy programs in a virtual machine. Linux seems to be an alternative to Windows, it is much stable than Windows, and it has better efficiency when it is used as the host system of multi-virtual machines. Not only that, but it is free. It also has an office suite which is named Libra Office.
