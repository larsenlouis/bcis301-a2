%!TEX root = ../assignment2.tex

\section{Data Analysis}

\subsection{Overview of Data}

Besides the interview scripts notes which was taken by Sarah last year, we also interviewed Sarah and Mehdi two weeks ago. Though Dabid is also planned to be interviewed by us, finally we didn't get a chance to interview him for unknown reasons. So those scripts and both interview to Sarah and Mehdi comprised our information resources. Based on this information, we conducted a qualitative analysis and got thirty-two facts across 12 functions.

\subsection{Functional Analysis}

Department of Computing is the department delicated teaching ICT students, which provides courses of software engeering, networking, business analysis, cyber security. Three divisions, that is administrative support, Operational Manager (ICT as a profession), and Operational Manager (Community Computing) comprise this Department. From the name of these divisionw we can see that two main teaching fields which are IT as profession and communication computing are provided in this department.

\subsection{Primary Themes}
Five primary themes of the issues were finally identified from thirty-one secondary themes in our coding data, which were budget, communication, manual process, security, and software. Below are a few typical samples for each themes.

\subsubsection{Budget} is about the expenditure for a department in annual reckoning. It is usually fixed but deficient since unexpected expense happened casually. Insufficient budget was noticed through the research at the first phase . The interviewee from CISCO Network Academy informed us that they need more expensive switches and routers to extend the network being used. They need more space to accommodate student and equipments. They also want to attract more students to learn networking, and they need more money to carry out marketing activities. They actually met difficulties in convincing Ara that paying more money to Cisco Network Academy is worthy enough. Budget issue is a realistic problem, while it is mainly a governance problem. Through deploying more Linux system can alleviate the budget issues to some extent, mostly, we cannot do more to resolve them by technical means. Therefore, we will not discuss this issue much in this article.

\subsubsection{Communication} has been recognized as a general issue for CISCO Network Academy and the Operations Manager. According the interview to the tutor of CISCO Network Academy, the desire to maintain a better connection with previous students was expressed. He also pointed out that nothing was taught on teaching students how to create connections with industry except the "geeky stuff" in their Academy. For our students, it is important to own soft power besides their technical power to get competitive power in the talents market. How to create a safe and effective social community? It is one of the important issue to be resolved. The Operation Manager claimed that "no one know we are the biggest CISCO system trainer in New Zealand". In Regards to the roles and responsibilities, there is a serious issue of lack of awareness and communication. Besides this complain, the engagment with student was also defective, it needs to be improved by better communication tool and mechanics.

\subsubsection{Manual process} is one of the most complained issue in this analysis. There are so many manual process involved in almost all functions in the Computing department. It was reported by all the interviewee, that is Mehdi, Sarah, the tutor from Ethics, the tutor from Tech Labs, the tutor from software engineering, and the OP Manager. All of them have met various inconvenience from manual process (mainly paperworks) in their daily work. Below are the details which are cataloged in functions for these inconvenience.

\paragraph{Administration} Mehdi, our dear leader, as the most important person in the Computing department, also feels pain on his administrative things since the internal process is manual. He pointed out that changing something in course, such as network protocol, he had to fill in forms and send the forms to corresponding division, then wait for approvals. All the processes, including the monitoring process, are manual process, it's lots of paperworks. As spectators, we can feel the pain when we review the interview scripts. Additionaly, the tutor from Tech Labs told us that the purchase management is also paper based. When they need to buy equipment or something else, they have to fill out quite a few paper forms for the IT department to review and purchase.

\paragraph{Timetabling} The interviewee which is the OP manager claimed that there is no specific IT program for timetabling, she has to talk with individual tutors and manually find out which teaching sessions or hours are required, and manually put this information into a spreadsheet. Even more distressing is that this is not the end. Once she complete the task on the spreadsheet,she has to manually put the timetable data into tribal software. She is tired with these things, it is not only boring, inefficient, but also error-prone. 

\paragraph{CAPEX Requests} Similar scene occurs on processing CAPEX requests. The same OP Manager is also in charge of processing CAPEX requests for computing team. Mannually filling in forms is required. Manually tracking the status of each CAPEX requests is an impossible task. Thus, there is no way to guarantee if the money has been correctly spent. 

\paragraph{Attendance} We have noticed this problem by ourselves. There is no an unified automatic attendance system in the Computing department. Part of teachers use class time to manually tick every student using the tribal application, which is rather time consuming in a larger class. Other teachers use a papersheet, which allows students sign on it. Sometimes a teacher or a teaching assistant may forget to pass the papersheet to students, which can cause all the students being "absent". This situation has been observed and experienced several times by us in this semester. In fact, a conclusion can be decided that both approach are far to ideal. The tutor from the Software Engineering section told us that they have tried many alternatives and didn't get an ideal one yet.

\paragraph{Result System} In regards to the results of the students, once a tutor confirmed the result for certain student, he or she could immediately get his or her result. It is good. However, in case the tutor makes a mistake with certain result, the tutor has to perform a manual process to change the result. 

\paragraph{Ethics Applications} The ethics tutor, who is charged with looking after the ethics in the computing department, he claimed that a student who needs ethics clearance has to fill out four ethics applications, and every application is a paperwork. Filling out the paper form is not conducive to keeping the participants anonymous. It is much better to fill in an online form remotely than a paper form.

\paragraph{Booking Classrom} Besides the functions we mentioned in earlier paragraph, Sarah, one of the teaching assistant representatives, she told us there's no any IT system to help them accomplish their assistant work. As a busy high grade student, she has to filled in paper timesheet to get paid. In addition, booking a classroom is also manual process, she has to talk to the Opearting Manager to get or confirm a classrom.



- Results Meeting:
     Used to identify students that may need additional Support
     Check consistency of grading
     Not a formal cross-checking process
     Preparation for the results meeting:
       No auto extraction from the Tribal Database
       Results are manually populated into a spreadsheet
       This spreadsheet is then discussed at the results meeting
       Occurs at the end of every semester prior to results being confirmed as final"
OP Manager	"Re-Enrolment – System generated form that is required to be printed and signed by staff/students
- All manual with forms – no System
   Would like for it to be more automated"
OP Manager	"Change management processes within the department have not been going well due to lack of resources
  - Groups of lecturers from every group get together to formally discuss and evaluate new curses
      Forms need to be filled out
      All paper based/MS word
      No application used to simplify the process
  - Institution doesn’t require lecturers to re-evaluate courses, but most do anyway
      Not a formal process
      Each program has a MOE required review
      Not at a course level, but at a program level (IE degree)
  - Academic services assist with the documentation side of things
      Not assisted by an IT system"
OP Manager	"Moderation of assessments
  - Decided to have a 5 year strategy
  - Each 5 yrs each course has an external assessment
  - Priority is given to new courses/tutors
  - Completely manual process
  - All in a spreadsheet
  - Someone has to remember to do it
  - Manual follow up if this has been done
  - Would like to try and minimalise reliance on actual people 
      Risk reduction
      Tech could assist"




\subsubsection{Security} 

\paragraph{Separated Networks}
Nowadays, networking technique develops fast day by day. However, for historical and security reasons, we were told that the network in CISO Network Academy is physically separated from the Ara network. The separation of the two networks is important for security. On the other hand, it has brought much inconvenience to teachers and students. For instance, somethings can only do on tech labs network, If a student need to access any content on the Ara network, he has to walk to an Ara computer labs to get those content then come back while the two network is separated. At the same time, a interviewee from Software Engineering division also complained that they have no permission to create virtual machines and cannot do certain things, such as install new version of the development tools. People know that this is because security reasons but inconvenience is inconvenience, there should be an approach can reach both the requirements from CISCO Network Academy and Software Engineering. In terms of current status, both networks have security problems. According one of the interviewees from Cisco Network Acddemy, around 50 pages security reports are generated for both tech labs netowrk and Ara network annually, and the issues in the reports needs to be fixed as soon as possible.

Tech Labs Tutor	"- Network is physically separate to the Ara Network
    Actually separate machines/server/internet connection
- Has two machines in his office – 1 x Ara, 1 x tech lans
    Can only access S Drive on the Ara machine
    Therefore both computers always need to be running"	the design of network reduce efficience	Security	Tech labs network access


\subsubsection{Software} 

\paragraph{Network Virtualization}


CISCO Network Academy	"Looking to move away from purchasing hardware, to cloud based networks
  Virtualised devices"	expect to have virtualized environment	Software	Cisco Program delivery
Software Engineering Tutor	"Looking at Docker
- Ike a sandbox environment
- Easily create and destroy programs


- Don’t have access to Docker on the Ara network
- Trying to work with Tech lab to run Docker on their network in the future
"	more freedom to run software	Software	Course delivery

\paragraph{Moodle and Myday}

Tech Labs Tutor	"- Moodle is an imperfect tool
     Terrible text editor – have to use HTML to make it look better
     Labs ect. Can be put into moodle
     Can log in and see what students have ben accessing
     Can identify students dropping the ball
- No notifications are given, have to physically check logs, but can then see what students have been accessing"	disadvantages of Moodle	Software	Administration
Tech Labs Tutor	"Licencing of software not always clear on virtual machines
Sarah	"
-	Moodle
-	Some parts are good - well covered
-	Some parts are not consistent - nothing really there
-	Task due day not working ( 10 years old due day )
-	Myday.ara.ac.nz is not used at all.
-	Directly use Moodle and outlook as alternative"	Disadvantages of Moodle / myday	Software	course delivery

\paragraph{Microsoft and Linux}

- Microsoft licencing in particular complex
     Have to be a lawyer and a genius to understand
     Took a month or two in the states to work out the licencing issue
     Always want to be legal, but need to be sure – Microsoft NZ couldn’t answer the question
     Remote access licences not valid – according to Microsoft
       Have LIBRA office instead
       Possibly looking at changing to LINUX desktops"	legal issues with Microsoft	Software	Course delivery
Mehdi	"-	Computing VS IT

