%!TEX root = ../assignment2.tex


\section{Introduction}

In this article, we seek to develop the 3-year IT deployment Plan for the Department of Computing of Ara Institute of Canterbury.

\subsection{Background}
% Ara 学校 介绍
Ara Institute of Canterbury is the biggest vocational training institute, which provides world-class, tertiary-level education throughout the Canterbury and Waitaki region. It provides 254 qualifications according to the New Zealand Qualification Authority(NZQA).

% Computing Department 介绍
Department of Computing is one of the departments, dedicated to information, communication and technology. The department offers various programs targeting programming, networking and information system.


\subsection{The Goal of the Deployment Plan}
% 提出 3 年 goal
Given the background of the department and the institute, the article seeks to produce a 3-year IT deployment plan with the ultimate goal of "teach better, study better".


\subsection{Topics and Structure}
To achieve the goal, research is conducted by interviews with staff, analysis of interview, proposals of solutions, evaluation of proposals, building the hierarchy of deployment plan.

To express the research in an organized way, the following structure is built to further all the topics.

In the introduction section, a general introduction is given, which covers the description of the institution, the department and how the article tries to research.

In methodology, the procedures of researching steps are defined, which covers how the data collected, how the analysis on data is conducted, how the potential solutions are made, how the solutions are evaluated, and eventually how the complete hierarchy of planning is delivered.

In data analysis, the comprehensive analysis is illustrated with the aforementioned method in the previous section. Sub-topics are given and issues are listed with detailed samples in each sub-topic. Then, one of the generic IT deployment strategies is chosen which fits the organization among 6 general types. The strategy chosen will be one factor for the proposals of solutions in the next section.

In solution analysis, proposals of solutions are offered coherently and cohesively to each sub-topic in the previous sections accordingly. An evaluation defined in the methodology is conducted for selecting the best solutions that benefit the organization.

In the deployment plan, a hierarchically structured deployment plan is delivered, which includes all the building elements in the relevant sections above.

Finally, a summary is given to conclude the result of this article.
% % 总起
% In this article, we seek to identify critical success factors for IT projects by literature review, and then try to build an evaluation system to make an IT project successful. With that aim, we conduct a qualitative analysis to gather all the significant points of what the literature argue regarding this topic. A quantitative model is set up to select and prioritize the findings. The outcome of the model is a solution list. The aforementioned process will be done twice with respective resources, which results in an evaluation tool for researchers and practitioners to avoid similar failures in IT projects.

% % 背景
% \subsection{Background}
% IT projects today tend to be agile, technical and complex\parencite[p. 2]{4} where pitfalls, issues and risks are everywhere in terms of time, people and budget. From a
% project management perspective, the project was 193\% over schedule, 419\% over budget, and 130\% over scope\parencite[p. 8]{6}. A shockingly high failure rate also can be seen across some other publications in literature\parencite{2,3}.

% % 背景中的三个关键词
% \subsection{Keywords}
% We investigate the literature and zoom in with the following keywords that influence success or failure of an IT project the most, which are project management, change management and risk management.

% \paragraph{Project Management}
% Project management is the planning to deliver the deliverables in a project, which academia believes, consists of 4-phase planning: initiating, planning, executing and controlling, while some researchers introduce the idea that this deliverable-targeted plan is considered as initiation, contagion, control and integration when a specific model in ERP projects is built\parencite[p. 3]{2}. Project management covers a lot of sub-areas such as time, scope, budget, quality, issues, risk and change. Success will be presented when a sophisticated project manager arrange them well in every stage.

% \paragraph{Change Management}
% Change management, as mentioned previously, is a sub-component of project management which researchers\parencite{3,6} have been putting emphasis on. In such an agile and technical and complex industry, the complexity of a project has inevitably made decision makers implement some changes for all or some stakeholders to transit to another new system\parencite[p. 1]{3} so that a better quality of the deliverables will be produced with the three constraints: scope, time, budget. The idea of change management has been regarded as a progressive method for attaining strong-minded conversion within a business or individuals sometimes\parencite[p. 3]{3}.

% \paragraph{Risk Management}
% Despite the fact that risk management is yet another sub-component of project management, a recent research by Bunker enlightens us by introducing disaster management from a change management perspective instead of a traditional project management perspective\parencite[p. 10]{6}. Intriguingly and similarly, the idea of emergent and scenario driven disaster management can be applied to the risk management.

% % 文章结构
% \subsection{Structure}
% The article is divided into 7 sections: introduction, methodology, findings of Phase 1, findings of Phase 2, conclusion, summary and reflections. In introduction, a brief information is given, stating what we are investigating and what result we will produce. 3 aforementioned keywords are also introduced for future reference. The methodology describes how we conduct our research. A subsequent findings of Phase 1 will be presented with details using the defined method with data, where an evaluation is done. The findings of Phase 2 is structurally similar to the previous section, but with 10 case studies. In outcome, we will compare and contrast findings of both phases, then try to make an evaluation tool from our findings. The conclusion will revisit what we will have covered in previous sections. Summary will be provided as a briefing of the whole article. And eventually, there is a reflections section discussing the values and limitations of this research.
