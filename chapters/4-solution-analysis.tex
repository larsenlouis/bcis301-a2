%!TEX root = ../assignment2.tex

\section{Solution Analysis}

In this section, some potential solutions are first put forward to tackle the identified issues in the previous section. Considered the generic IT deployment strategy of the organization, the solutions are topic based and then evaluated as defined in methodology. The outcome of this section is the production of the selected solutions to be put in the hierarchy of planning.

\subsection{Potential Solutions}

According to the methodology, the following possible solutions are drafted to solve the topics of issues that have discussed in the previous section. Given the evaluation of the generic IT deployment strategy, the computing department is considered to fit in the category of "centrally planned". Therefore, there will some solutions trying to handle the 5 topics of issues identified in data analysis respectively.

\subsubsection{Linux with Docker}
Linux with docker is a solution for issues related to budget, security, and software.

Linux is a free and opensource operating system started in 1991, which is a free version of UNIX system back in the 1970s. Linux is able to be running on multiple architectures with various configurations of hardware.

Docker is a container platform that is able to separate the environment of execution from the physical operating system running on the physical machine. Docker relies on hypervisor software such as VirtualBox, VMWare Workstation and so on. For less budget to be spent, the free and opensource Virtualbox is selected to run the docker on Linux platform.

The Combination of Linux and Docker is due to the following factors. Firstly, docker is more performant in Linux physical machine via VirtualBox abstracting and visualizing the hardware. Secondly, the current IT deployment involved with a centralized administration on Windows platform fails to deliver the freedom and the accessibility of installing and running students' software, especially the networking students who study the configuration of operating systems with tedious and repeated setups of virtual machines. Finally, according to the TechLab tutor, Linux is considered to be the operating system for their labs, which favors the need for both tutors and students.

\subsubsection{Telegram}

Telegram is an end-to-end encrypted instant messaging application running on multiple platforms such as Windows, MacOS, Linux, Android(4.1 and above), and iOS (8.0 and above), even the Windows Phone OS with so small number of users. In case a public purpose computer is used, for instance, in a computer Labs or in a library, which is not allowed to install any application, a few solution are provide in this situation. The web version or web app can be used. One of their famous sologan is that "A native app for every platform". The did it.

Firstly, Telegram provides the messaging functionalities that other IM applications fail to achieve. In Telegram, all the messages are tagged with a reading status of unread or read. This feature especially fits the enterprise working environment, targeted for more effective communications. Moreover, telegram is able to send and receive files of any type that are up to 1.5 GB in size each, the files which are just sent can be accessed instantly on their other devices. Most importantly, the files which are saved at Telegram, will not expire at a specific time. It can always be accessed by its users, whatever how many files are uploaded. It really is a reliable and secure instant messaging platform.

Secondly, Telegram groups are ideal for sharing stuff with friends and family or collaboration in small teams. However, Telegram groups can also grow very large and support communities of up to 200,000 members. Any group in Telegram can be made public, its persistent history is also can be toggled to control if new members can access the earlier messages. In Telegram group, appointing administrators with granular privileges is supported. In case a most important message need to be highlighted, pinning specific message at the top of a group function is provided to answer this question. The messages being pinned can be seen to all members including those who have just joined.

Thirdly, Telegram Channels are another powerful tool for broadcasting message to large audiences. In fact, a Telegram Channel can have unlimited subscribers. If a message is posted in a channel, it is singed with the channel's name and photo instead of the author's. Each message in a channel has a view counter that gets updated when the message is viewed, including its forwarded copies.

Finally, it is totally free. Unlike many of its competitors, Telegram does not mean to earn money. They believe in "fast and secure messaging that is also 100\% free".  In the foreseeable future, they have quite enough money to support their running. In case Telegram runs out, they will introduce non-essential paid options to support the infrastructure and to pay salaries to its developer. They promise that making money will never be an end-goal for Telegram.

Telegram brings the idea of 'agile' to the small and flexible working environment in the Department of Computing. Its powerful group feature offers unified history, all platform access, instant search, replies, mentions, and hashtags, pinned messages, file sharing, group permission, to name but a few. Those features just mentioned enable the department to solve kinds of communication issues and security concerns partially.

\subsubsection{Slack}
Slack is yet another solution to communication issues. It is an enterprise-oriented instant messaging application that features projects, channels, and other high customizable functions.

Another highlighted feature is its channels. Channel acts as a group chat on a specific topic. Slack enables permission administration on channels so that users of different readability are organized to carry out scheduled tasks accordingly. Another practical feature is the chatting history. A newly joined user is able to view the messages and files submitted to the channel before he or she joins.

Slack enhances the communication over the tools which have been used in the Department of Computing. It does not  mean to replace them,  but it can make them better. Slack has a much powerful ability to integrate with existing service. For instance, sharing files in Slack is easier, since it can find and share files from OneDrive, Dropbox, Google Drive etc without leaving Slack.

The drawback of Slack is that it is not free. It do provide a free plan, but many limitations are applied on that plan, this has weakened its competitiveness to some extent. 

\subsubsection{Jira}

Jira is a comprehensive tool for issue and tracking. Jira can reduce the manual process involved in the department and lower communication barriers. There remain so many manual processes in the workflows of several functions inside the department. Jira administrator is able to define customized workflows in a context connects several roles. Process is defined in a workflow chart that stakeholders are fully aware of the progress they are making. Permission management is well covered by scopes of project, group, user and workflow. In other words, a task goes through the workflow that can decide who does something at what time and that who has the permission to view or alter state or history. Moreover, Jira integrates a reporting system in the software that helps to do statistics for future adjustment in the department.

With careful investigation and design of workflows in the department, Jira is able to digitalize the communication process in various kinds of tasks and facilities high efficiency of communication.

\subsubsection{Windows 10 Sandbox}

Tutors and students always want to run their favorite program in computer labs. However, running executable files on Windows is a rather risky thing since the executable files which are downloaded from the internet are mostly not safe. It can be a serious issue to the Windows operating system. Fortunately, Microsoft has done a good job to solve that kind of issue. That is Windows Sandbox.

The idea is not fresh, which is running a windows instance in a virtual machine. The highlight is that you can run that instance without an extra license. Just use it, no worries of license are necessary to consider. 

As we have mentioned earlier, the Computing Department is experiencing the restriction of using specific software in labs. One of the reasons causing this is the whole dependency of the centrally governed administration policy with Active Directory. Nevertheless, this technology lowers the cost of management of rooms of machines. The number of installed programs prevents students from wanting to run the helpful tools they want. 

The new Windows 10 sandbox feature resolves this problem in a graceful manner. Impressively, the requirements to the host computer are rather low: Windows 10 Pro or Enterprise build 18301 or later, X64 architecture, 4GB RAM, 1GB free disk space, 2 CPU cores. The best part of this approach is that no virtual hard disk is needed. Instead, Windows dynamically generates a clean snapshot OS based on the Host OS on the current machine. This approach does not copy files which the VM needs but link them. It saves much disk space and runs faster. The VM image is incredibly small, just around 100 MB. As we have mentioned, since it is a copy of your OS, no separate license key is needed.  

Only a tiny effort of configuration to enable this feature and to set up some restriction can make a total difference of the deployment of software in computer labs. In addtion, a policy can be implemented to limit the disk usage of students and to grant them permission to run all kinds of software they wish.

\subsubsection{New Network and Servers}
The setup of the existing two separated networks ensures the security but accessibility is impacted. On the network, there are obsoleting services running that need changing for the centralized services of the local area network. Furthermore, the Tech Labs tutor demands new servers to scale up. We argue that there should be a new design of architecture that solves the security and accessibility problem at the same.

Site-to-site VPN is a method for connecting two isolated networks with controllable configuration over a firewall. New servers should be purchased to scale up the labs and the servers that powers the local SMB shared folders with redundancy.

\subsection{Evaluation of Solutions}
An evaluation of the aforementioned candidate solutions is done according to the framework mentioned in the Methodology section.

\subsubsection{Rating of Candidate Solutions}

Scores are given to the topics of issues in the section of Data Analysis, namely budget, communication, manual process, security, and software. The result is shown in \autoref{tab:eva}.

\begin{table}[!ht]
\caption{Ratings of Candidate Solutions}
\begin{adjustbox}{width=1\textwidth}
\csvautotabular{tables/eva.csv}
\end{adjustbox}
\label{tab:eva}
\end{table}

\subsubsection{The Result of Evaluation}
The average score of all the candidate solutions $\bar{R_{S}}$ is 17.833. According to the aforementioned evaluation framework in Methodology, The following solutions are eligible to become the final solutions for this IT deployment plan of the Computing Department of Ara Institute of Canterbury. They are Jira, Telegram, Linux deployment with docker, and Windows 10 Sandbox. With further inspection of the selected solution, we find that all the 4 solutions cover all the topics of issues.


% \subsection{Overview of Data}
% In Phase 2, 10 case studies are given on how projects failed in Victoria Australia. The cases are government-related information communication and technology projects. The author\parencite{case_study} provides well-structured anatomy of projects where issues are discussed in depth by topics. Additionally, some recommendations are given by the author, some of which is responded by the project owners.

% After the qualitative analysis, the same primary themes are used bacause we have found they can be appiled in this research context without much loss of information and it will possibly reduce the complexity of the evaluation tool we are trying to build. 

% \subsection{Gaps and Solutions}
% Using the same method, we collect gaps and possible solutions(proposals) for further evaluation. There are 54 gaps reflecting 54 proposals to 10 case studies, some of which are duplicate in meaning. The author uses no mathematical model to prioritize gaps, a baseline value of $1$ is also used to all the proposals, according to the predefined methodology.

% The collected data are tagged with the same four themes: control, process, people and structure. Here are some samples under each theme.

% \subsubsection{Samples: Control}
% % id 21
% In case 4, the author argues that the longer planning that spans in time, the more risk of redesigned. Ultimately, the project will be likely to fail due to this uncertainty of change. Project leaders should plan to fund well with top management support in the initial phase. As change happens at any time, if the project cycle is too long, the project will be more likely to require more funding. Correspondingly, the uncertainty of the project increases with time, and the initial design and costing plan are very likely to need to change over time, so the risk of project failure increases.


% % id 2
% In case 1, it is the rushing to meet deadlines that fails to deliver design goals. For cover-up, benefits are altered to please the government. Project managers should define the measurement of deliverables and change accordingly. A good case plan is the foundation of a project's success. If the case plan is just made to win the government support without respecting the facts, the result of project failure is foreseeable. In this case, many of benefits are unmeasurable but they are still written into the plan to please the government. Its failure is unavoidable.

% \subsubsection{Samples: Process}
% % id 26
% Poor user training and post-implementation support result in bad experience among users. Better user training and support should be designed and implemented in the early stage of project planning. In case 5, even the project (CRIS) has been delivered to the end-users in July 2008, because of the lack of necessary training and the poor system support services, the feature change request from the user cannot get a proper response in time, which results in pretty low user satisfaction of the system.

% % 30
% The training cost is not included in the planning process. It is suggested to consider training in cost planning. In case 6, due to lack of adequate up-front planning, the project plan didn't include the expense for Ultranet coaches and the cost for school professional development day. Thus it caused a funding problem around \$23 million. A comprehensive plan is the cornerstone of the project's success, to avoid a funding gap, IT projects should be handled with care when reviewing funding plans.

% \subsubsection{Samples: People}
% % 34
% Failing to detect risks of vendor results in bad contract deals. Vendor performance should be reviewed by a third party in planning. In case 7, even DOJ has made various efforts to address vendor performance issues, the vendor is still frequently failing to meet the promised timeline. In September 2008, the owner of the vendor changed, and DOJ had to renegotiate the contract with the new owner of the vendor. Due to fearing that the new owner would abandon the contract, DOJ had to make concessions on the liability for delay and the indemnity clause. This fully demonstrates the importance of an objective assessment of the supplier's performance capabilities. It is recommended that the tenderer should hire an independent, qualified third-party agency to evaluate the vendor's capabilities and historical performance.


% % 37
% Projects fail and lead to an alternative solution when requirements are not communicated precisely. It is important to understand the requirements and define the deliverables that the project manager and the end-user agree on.

% According to the Supreme Court of Australia, ICMS’s case management system (CourtView) fails to meet the court’s needs. The Supreme Court has ultimately resolved to pilot its own system to provide case management. ICMS is a typical case which is developed by the client and vendor. The end user did not get an opportunity to participate in their opinion before the system is shipped to them. Systems developed in this way are very likely to cause dissatisfaction to end users. After all, most of the time, the people who pay (the customers or clients) are not the people who use the system. Not surprisingly, the Supreme Court finally chose to fund the development of the system it needed, spending only a small amount of money and getting a good satisfaction.

% \subsubsection{Samples: Structure}
% % 4
% The vacancy of a project leader results in management chaos. It is important to have a project leader. Unfortunately, they appointed two project manager in case 1, one is a business project manager, the other is a technical project manager. Since none of them is authoritative, neither person is responsible for the overall result of the project. To make matters worse, after-the-fact investigations have shown that both project managers are lack experience in managing large IT projects. A qualified, authoritative is irreplaceable for such projects.

% % 19
% The organization structure stops from making timely funding to implement the project. A better organization is advised. There are uncertainties in any IT project, and good organizational and funding plans can help overcome these uncertainties. The VicRoads project did not receive follow-up development funds, as a result, the development team has basically been disbanded, and the project prospects are not good. VicRoads has already spent \$52 million in past years, it will be a huge waste to Australia government. 


% \subsection{Evaluation of Solutions}
% To select the solutions out of all the proposals, we do the same evaluation steps as shown in \ref{section:evaluation}.

% With the defined method, we tag all our research contexts as $C_{1}, C_{2}, \ldots, C_{10}$. We assume all the solutions from the 10 case studies are of equal importance, given the fact that there is no math model used in Phase 2. Therefore, we pad the cross-contextual coefficient with a baseline value of 1 to all the $\mathit{k_{C_{i},j}}$ in formula \ref{brief}. So formula \ref{final} can be simplified as 

% \begin{equation}
% g_{C_{i},j} = l_{C_{i},j} = |P_{C_{i,j}}| |R_{C_{i,j}}|
% \label{formula:p2}
% \end{equation}

% With the updated formula \ref{formula:p2}, we calculate the global influence score for each proposal. Here is a visualization of the scores of all the proposals in Phase 2.

% In Figure \ref{fig:rankingp2}, there are 3 groups: the leading group of 3 proposals, the group of majority and the trailing group.
% \begin{figure}[!ht]
% \centering
% \caption{Proposal Ranking in Phase 2}
% \resizebox{\columnwidth}{!}{%
% \includegraphics[height=10em]{global_influence_score_p2.png}
% }
% \label{fig:rankingp2}
% \end{figure}


% \subsection{Result of Evaluation}
% We calculate the average score using the same way($\bar{g}=1.37$), which truncates the trailing group. The selected proposals are shown as follows.

% \begin{table}[!ht]
% \caption{Solution List Of Phase 2}
% \begin{adjustbox}{width=1\textwidth}
% \csvautotabular{tables/solutions_p2.csv}
% \end{adjustbox}
% \label{tab:solution2}
% \end{table}
